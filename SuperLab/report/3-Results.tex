\section{Description des test}
5 bancs de test ont été définis pour tester différents comportement du XF.
Ces 5 bancs de tests sont disponibles sous Qt, et sous STM32 avec STM32CubeIDE.
\subsection{Test-bench 1}
Le premier test instancie 2 machines d'états générant des timeouts, puis
transmettant une phrase à chaque timeout : un premier transmet "Say Hello" chaque
seconde, le second: "Echo" chaque 500ms.

\subsection{Test-bench 2}
Le second test instancie deux machines d'état identique, à l'exception qu'une
est un objet dynamique (et donc destructible).
Les deux objets vont passé un décompte de 5, avant d'être terminé. L'objet dynamique
devra être détruit et son destructeur devrait afficher "Called destructor".

\subsection{Test-bench 3}
Ce test check la gestion basique des événement dans les machines d'état.
Une machine d'état s'envoie périodiquement à elle même un "evRestart" pour
passer d'un état à l'autre.

\subsection{Test-bench 4}
Génère plusieurs timeout, puis crée un restart, qui devra unschedule des timeouts.
Si ce test n'est pas réussi, la liste des timeouts devrait se remplir, et on
aurait des comportements étranges.

\subsection{Test-bench 5}
Transmets plusieurs timeout en même temps au Timeout Manager pour stress-test l'ajout
intelligent de timeouts.

\section{Discussion des résultats}
Tous les tests ont été validés sur les deux systèmes. Le timing du système embarqué
est compliqué à mesurer. Nous utilisons le port série pour communiquer les
textes au PC, qui va stocker dans un buffer les données et les sortir quand il
pourra, donnant des timings étant parfois 20ms en avance, des fois 20ms en retard.
Les timings fournis en Annexe B seront donc uniquement les timings mesurer avec Qt.